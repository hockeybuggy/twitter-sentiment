\documentclass{article}
\usepackage{float}
\usepackage[margin=1.5in]{geometry}
\usepackage{setspace}

\onehalfspacing

\begin{document}
\large
\thispagestyle{empty}

\section*{Project Title}

{\bf Using a maximum entropy classifier for Sentiment analysis of Twitter Data}

\quad

This project will be carried out by Douglas Anderson in the summer of 2014,
under the supervision of Professor Fei Song.

\section*{Subject Description and Expected Results}

Determining public sentiment is a difficult task that can help Governments,
Companies, and Organizations make more informed decisions. In the modern era of
social media we have an opportunity to gauge people's feeling about events
mechanically using Natural Language Processing techniques.

The dataset and inspiration for the project is the provided by Task B outlined
in task 2 of the 2013 International Workshop on Sentiment Analysis
(http://www.cs.york.ac.uk/semeval-2013/task2/). The goal is to create a
classifier (or set of classifiers) that will reliably determine the message
polarity of a Twitter Message (a `tweet'). The constrained nature of tweets
lead to abbreviations and their informality leads to incorrect spelling and
grammar. These issues can add difficulty to the classification process.

The maximum entropy modelling is a machine learning approach that has, in
previous work, performed well in loosely structure text. Over the course of the
semester Douglas will design, implement, test and report upon possible
improvements on existing classification methods.

\section*{Evaluation Method}

\begin{itemize}
    \item Literature survey: 15\%
    \item Discussion, analysis ad design: 30\%
    \item Implementation, testing and experimental results: 30\%
    \item Final oral presentation and written synthesis: 25\%
\end{itemize}

\end{document}

